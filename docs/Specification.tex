\documentclass[oneside,german,oneside]{scrbook}
\usepackage{ae,aecompl}
\usepackage[utf8]{inputenc}
\usepackage[ngerman]{babel}
\usepackage{graphicx}
\usepackage{caption}
\usepackage{dcolumn}

% Outline numbering
\setcounter{secnumdepth}{0}
% Fette Spalten
\newcolumntype{b}{>{\bf}l}

\title{aidGer Spezifikation}
\author{Buchgraber (2512046), Gildein (2513744), Pirrung (2526016)\\
Gruppe 10}
\date{\today}

\begin{document}

\subject{- vertrauliches Dokument -}

\maketitle
\pagebreak

\tableofcontents
\pagebreak

\section{Einleitung}\label{sec:Einleitung}

    \subsection{Zweck}\label{sec:Zweck}

        Die Spezifikation dient als Grundlage f\"ur alle Dokumente, die w\"ahrend
        des hier beschriebenen Softwareprojekts entstehen. Sie enth\"alt alle
        wesentlichen Anforderungen an die Software und deren Schnittstellen.

    \subsection{Leserkreis}\label{sec:Leserkreis}

        Dieses Dokument ist f\"ur den folgenden Leserkreis bestimmt:

        \begin{itemize}
          \item das gesamte Projektteam
          \item den Kunden
          \item die Dozenten/Betreuer
          \item k\"unftige Programmierer/Verwalter dieses Projektes
        \end{itemize}

    \subsection{Einsatzbereich und Ziele}\label{sec:Einsatzbereich}

        In diesem Projekt soll das Hilfskräfteverwaltungssystem
        ``aidGer'' realisiert werden. Der ``aidGer'' soll den Mitarbeitern
        der Abteilung SE die tägliche Arbeit bei der Verwaltung
        der Hilfskraftbeschäftigungen des Institutsverbund der Informatik
        erleichtern soll. \\
        Die neue Software dient grundsätzlich zur Erleichterung der 
        Erfassung und Auswertung aller Prozesse, die bei der Verwaltung 
        von Hilfskräften anfallen. \\
        Das Ziel ist es, mit ``aidGer'' die jetzige Lösung ``Hive'' zu 
        ersetzen. Sie soll jedoch weiterhin einsetzbar sein, um im Notfall
        auf alte Prozesse zurückgreifen zu können.

    \subsection{Fachbegriffe und Abk\"urzungen}\label{sec:Fachbegriffe}

        Alle f\"ur dieses Projekt relevanten Fachbegriffe und 
        Abk\"urzungen sind im \ref{sec:Begriffslexikon} (im Anhang 
        dieses Dokuments) aufgef\"uhrt.

    \subsection{Aufbau dieses Dokuments}\label{sec:Aufbau}

        Neben einer allgemeinen Beschreibung des Systems sollen
        die Anforderungen an die Funktionen des Systems und die geforderten
        Qualit\"aten hinsichtlich der Software selbst dokumentiert werden.

    %TODO: Issue #6
    %\subsection{Einschr\"ankungen}\label{sec:Einschraenkungen}

    \pagebreak

\section{Allgemeine Beschreibung}\label{sec:AllgemeineBeschreibung}

    In diesem Kapitel soll der ``aidGer'' in seinen Grundz\"ugen beschrieben
    werden.

    \subsection{Einbettung}\label{sec:Einbettung}
    
        %TODO: Issue #5

        Der ``aidGer'' soll übersichtlich und einfach zu bedienen sein 
        um eine möglichst gerine Einarbeitungszeit zu ermöglichen. Es soll
        auf möglichst vielen Betriebssystemen lauffähig sein.

    \subsection{Grundlegende Funktionen}\label{sec:GrundlegendeFunktionen}

        Der ``aidGer'' soll dem Nutzer folgende Grundfunktionen
        bereitstellen:

        \begin{itemize}
          \item Hinzuf\"ugen, \"andern und L\"oschen von Filmen
          \item Auflisten aller vorhandenen Filme in tabellarischer Form
          \item Freitext-Suche zum Finden von Filmen
          \item Betrachten der Details eines Filmes
          \item Verleihen und Zur\"uckgeben von Filmen
          \item Anzeige verliehener Filme nach Verleihdauer
          \item Export der Filmsammlung im HTML-Format
        \end{itemize}

        Diese Funktionen sollen f\"ur alle Benutzer leicht erlernbar, effizient
        und einfach in der Handhabung sein.

    \subsection{Annahmen und Abh\"angigkeiten}\label{sec:Annahmen}

        Bei der Spezifikation wurde von folgenden externen Einflussfaktoren
        ausgegangen:

        \begin{itemize}
          \item Da von unbezahlten Hilfskr\"aften ausgef\"uhrt, besteht kein
          Geldmangel.
          \item Das System soll eine m\"oglichst lange Lebensdauer haben und muss
          deshalb m\"oglichst flexibel sein.
          \item Es stehen gen\"ugend Filmbesitzer als Tester zur Verf\"ugung
        \end{itemize}

    \subsection{Entwurfseinschr\"ankungen}\label{sec:Entwurfseinschraenkungen}

        Die Vorgabe des Kunden hinsichtlich der Entwicklungsplattform ist
        Java  mit der SDK Version 6 als Programmiersprache und Swing als 
        Oberflächenbibliothek. \\
        Zur Generierung von PDF Dateien soll eine aktuelle Version der
        iText Bibliothek eingesetzt werden. Um auf die Datenbank 
        zuzugreifen, soll die von Team AdoHive bereitgestellte AdoHive 
        Bibliothek verwendet werden. Andere externe Bibliotheken
        müssen durch den Betreuer explizit genehmigt werden.\\ \\
        Zur Darstellung von Klassendiagrammen oder ähnlichen Diagrammen,
        die während des Entwurfs entstehen, soll UML benutzt werden.

    \subsection{Externe Schnittstellen}\label{sec:Schnittstellen}

        Der ``aidGer'' benötigt eine Schnittstelle zur verwendeten 
        Datenbank. Diese wird durch die Bibliothek AdoHive zur Verfügung
        gestellt. Zudem wird eine Verbindung mit dem bereits vorhandenen
        Ticketsystem benötigt.

    \pagebreak

\section{Anforderungen an die Funktion}

    In diesem Kapitel werden die funktionalen Anforderungen an das System im
    Detail beschrieben.

    \subsection{Leistungsanforderungen}\label{sec:Leistungsanforderungen}

        Da das Programm auf dem Computer des Benutzers verwendet wird, sollte es
        andere Programme oder das Betriebssystem nicht behindern und alle
        Funktionen in m\"oglichst geringer Zeit ($<$ 2 Sekunden) ausf\"uhren.\\
        Ausnahmen k\"onnen hierbei der Export oder das Abspeichern sehr grosser
        Sammlungen darstellen.

    \subsection{Mengenger\"ust}\label{sec:Mengengeruest}

        Folgende Kenngr\"o{\ss}en sind f\"ur das System relevant:

        \begin{itemize}
          \item Es wird von ca. 100 Filmen pro Sammlung ausgegangen. Mehr
          abzuspeichern sollte jedoch auch kein Problem darstellen.
          \item Pro Film werden 10 Datenpunkte abgespeichert
          \item Wird oder war ein Film verliehen, kommen 3 weitere Datenpunkte
          hinzu. (Sollten jedes Verleihen eines Filmes abgespeichert werden
          vervielfacht sich diese Zahl nat\"urlich (OPTIONAL))
        \end{itemize}

    \subsection{Anforderung A-00: Film hinzuf\"ugen}\label{anf:00}

        Der ``aidGer'' soll dem Nutzer eine simple M\"oglichkeit bieten, die
        Daten eines Filmes einzutragen. Vor dem Speichern der Daten wird
        \"uberpr\"uft ob alle ben\"otigten Informationen eingetragen wurden. Zu
        diesen Daten geh\"oren jeweils:

        \begin{itemize}
          \item Filmtitel und Untertitel
          \item Erscheinungsjahr
          \item Genre
          \item Inhaltsbeschreibung
          \item Medium
          \item Sprache(n)
          \item L\"ange (in Minuten)
          \item Pers\"onliche Bewertung auf einer Skala von 1 bis 5 Sternen
          \item Kaufdatum
        \end{itemize}

        Es soll zudem m\"oglich sein neue Genres anzulegen (M\"oglicherweise auch
        neue Medien anlegen f\"ur maximale Flexibilit\"at (OPTIONAL)).

    \subsection{Anforderung A-01: Film bearbeiten}\label{anf:01}

     Die Informationen bereits hinzugefügter Filme sollen einfach editiert werden
     können. Dabei kann derselbe Dialog wie beim Hinzufügen von Filmen (siehe
     \ref{anf:00} verwendet werden. Jedoch sollten bereits vorhandene
     Informationen bereits eingefügt sein.

    \subsection{Anforderung A-02: Film l\"oschen}\label{anf:02}

        \"uber einen Knopfdruck sollte das L\"oschen eines Filmes m\"oglich sein. Vor
        der Durchf\"uhrung sollte der Benutzer noch einmal gefragt werden ob er
        sich sicher ist.

    \subsection{Anforderung A-03: Anzeige der Filmsammlung}\label{anf:03}

        Die Filmsammlung soll in einer Tabelle mit den Spalten ``Titel'',
        ``Erscheinungsjahr'', ``Genre'', ``Medium'' und ``Bewertung'' angezeigt
        werden. \"uber die Spalten soll sich die Sammlung sortieren lassen.

    \subsection{Anforderung A-04: Anzeige des Umfangs der Sammlung}\label{anf:04}

        Der Umfang der Filmsammlung (die Anzahl der Filme und die Gesamtl\"ange
        aller Filme) sollte in der Statusleiste jederzeit einsehbar sein. Im
        Falle einer stattfindenden Freitext-Suche (siehe \ref{anf:05}) sollten
        sich die Werte dem Ergebnis der Suche anpassen.

    \subsection{Anforderung A-05: Freitext-Suche}\label{anf:05}

        Mit Hilfe eines Suchfeldes soll die Suche nach bestimmten Filmen m\"oglich
        sein. Dabei wird ohne R\"ucksicht auf Gro{\ss}- und Kleinschreibung in den
        Feldern ``Titel'', ``Untertitel'' und ``Inhaltsbeschreibung'' gesucht.
        Die gefundenen Filme sollen in einer Trefferliste angezeigt werden.

    \subsection{Anforderung A-06: Speichern der Sammlung}\label{anf:06}

        Die Filmsammlung soll automatisch beim Verlassen der Anwendung
        gespeichert werden. Der Benutzer soll jedoch auch selbst abspeichern
        k\"onnen. \\
        Der Speicherort kann von der Anwendung vorgegeben werden.

    \subsection{Anforderung A-07: Verleihen von Filmen}\label{anf:07}

        Die Filme der Sammlung sollen an andere Leute verliehen werden und von
        diesen zur\"uckgegeben werden k\"onnen. Dazu werden f\"ur jeden Verleih
        folgende Daten gespeichert:

        \begin{itemize}
          \item Film (ein Verweis auf den Film in der Sammlung)
          \item Ausleihender
          \item Verleihdatum
          \item R\"uckgabedatum
        \end{itemize}

        Derzeit nicht verliehene Filme k\"onnen unter Angabe des Ausleihenden und
        des Verleihdatums verliehen werden.\\
        Diese Filme sollen in der Anzeige der Sammlung farblich hervor gehoben
        werden oder eine andere entsprechende Markierung besitzen.\\
        F\"ur jeden Film soll mindestens der letzte oder momentane Verleih
        angezeigt werden.

    \subsection{Anforderung A-08: Filterfunktion f\"ur verliehene
    Filme}\label{anf:08}

        Mithilfe einer Filterfunktion sollen alle verliehenen Filme angezeigt
        werden k\"onnen die l\"anger als einen bestimmten Zeitraum verliehen sind.
        Dieser Zeitraum soll frei eingestellt werden k\"onnen.

    \subsection{Anforderung A-09: HTML-Export der Sammlung}\label{anf:09}

        Die komplette Filmsammlung soll sich in ein HTML-Dokument exportieren
        lassen. Dabei sollen alle Daten (au{\ss}er Verleihinformationen) angezeigt
        werden k\"onnen. Der Speicherort soll vom Benutzer frei w\"ahlbar sein.

    \subsection{Anforderung A-10: Format/Details der
    Implementierung}\label{anf:10}

        Die GUI muss in Swing implementiert werden und soll Deutsch oder
        Englisch sein.\\
        Der Code soll gem\"a{\ss} den Programmierrichtlinien von Sun geschrieben
        werden und nach Javadoc-Konvention kommentiert werden.Das fertige
        Programm muss unter JDK 6 laufen. \\
        Die zentrale Klasse der Anwendung (mit der \textit{main()} Funktion)
        soll \textit{MovieManager} heißen

\pagebreak

\section{Benutzeroberfl\"ache}\label{sec:Benutzeroberflaeche}

    Die Benutzeroberfl\"ache des ``aidGer'' soll mit Swing erstellt werden
    und sich in das Erscheinungsbild des Betriebssystems einpassen. Die
    folgenden Mockups zeigen wie sie ungef\"ahr aussehen soll.

	  %\includegraphics[scale=0.5]{Mockup1.png}
  	%\captionof{figure}{Mockup}

  	\bigskip

  	Die Datentabelle soll dabei bei jeder Eingabe im Suchfeld dynamisch die
  	Anzeige dem Suchwort entsprechend ver\"andern. Ebenfalls \"andern sollen sich
  	dabei die Datenbankdetails und die Werte der aktuelle gefundenen Filme
  	anzeigen.

\pagebreak
\section{Anwendungsf\"alle}\label{sec:Anwendungsfaelle}

    In diesem Kapitel sollen die wichtigsten Anwendungsf\"alle des Systems
    dargestellt werden.
    
    \subsection{Use-Case-Diagramm}\label{sec:usecasediagramm}
    
    %\includegraphics[scale=0.5]{Anwendungsfaelle.png}
    %\captionof{figure}{Anwendungsf\"alle}

    \subsection{Anwendungsfall A-00}\label{uc:00}

      \begin{tabular}{|b|p{10cm}|}
        \hline
        Name & Film hinzuf\"ugen \\
        \hline
        Aktoren & Benutzer \\
        \hline
        \hline
        Vorbedingung & Applikation wurde gestartet, Benutzer befindet sich in
        der Hauptansicht.\\
        \hline
        Regul\"arer Ablauf & Benutzer klickt auf Schaltfl\"ache ``Film hinzuf\"ugen''
        und ein Dialog \"offnet sich. Der Benutzer gibt die Daten des Filmes ein
        und klickt auf ``Speichern''.\\
        \hline
        Nachbedingung & Der neue Film wurde intern abgespeichert und wird nun
        in der Tabelle angezeigt.\\
        \hline
        Alternative Abl\"aufe &
        \begin{enumerate}
          \item Der Benutzer klickt auf ``Abbrechen'', der Dialog schlie{\ss}t sich
          und keine Daten werden abgespeichert.
          \item Der Benutzer gibt inkorrekte Daten in den Dialog ein. Nach dem
          Klick auf ``Speichern'' gibt die Anwendung einen Fehler zur\"uck.
        \end{enumerate}
        \\
        \hline
      \end{tabular}

    \subsection{Anwendungsfall A-01}\label{uc:01}

      \begin{tabular}{|b|p{10cm}|}
        \hline
        Name & Film bearbeiten \\
        \hline
        Aktoren & Benutzer \\
        \hline
        \hline
        Vorbedingung & Benutzer befindet sich in der Hauptansicht und es
        sind Filme in der Sammlung vorhanden.\\
        \hline
        Regul\"arer Ablauf & Der Benutzer selektiert einen Film und klickt auf
        ``Bearbeiten''. Im sich \"offnenden Dialog editiert er die gew\"unschten
        Informationen und klickt auf ``Speichern'' um den Vorgang abzuschließen. \\
        \hline
        Nachbedingung & Die ge\"anderten Informationen des Filme werden gespeichert
        und in der Tabelle angezeigt.\\
        \hline
        Alternative Abl\"aufe &
        \begin{enumerate}
          \item Es ist kein Film selektiert. Der Benutzer wird darauf hingewiesen.
          \item Der Benutzer klickt auf ``Abbrechen'', der Dialog schlie{\ss}t sich
          und keine Daten werden abgespeichert.
          \item Der Benutzer gibt inkorrekte Daten in den Dialog ein. Nach dem
          Klick auf ``Speichern'' gibt die Anwendung einen Fehler zur\"uck.
        \end{enumerate}
        \\
        \hline
      \end{tabular}

    \subsection{Anwendungsfall A-02}\label{uc:02}

      \begin{tabular}{|b|p{10cm}|}
        \hline
        Name & Film l\"oschen \\
        \hline
        Aktoren & Benutzer \\
        \hline
        \hline
        Vorbedingung & Mindestens ein Film muss sich in der Datenbank befinden.
        Der Benutzer befindet sich in der Hauptansicht.\\
        \hline
        Regul\"arer Ablauf & Der Benutzer w\"ahlt einen Film in der Tabelle aus und
        klickt auf ``L\"oschen''.\\
        \hline
        Nachbedingung & Der Film wurde aus der Datenbank gel\"oscht und ist in
        der Tabelle nicht mehr sichtbar.\\
        \hline
        Alternative Abl\"aufe &
        \begin{enumerate}
          \item Es ist momentan kein Film ausgew\"ahlt. Der Benutzer wird darauf
          hingewiesen.
        \end{enumerate}
        \\
        \hline
      \end{tabular}

    \subsection{Anwendungsfall A-05}\label{uc:05}

      \begin{tabular}{|b|p{10cm}|}
        \hline
        Name & Freitext-Suche \\
        \hline
        Aktoren & Benutzer \\
        \hline
        \hline
        Vorbedingung & Der Benutzer befindet sich in der Hauptansicht und es
        sind mehrere Filme in der Datenbank vorhanden.\\
        \hline
        Regul\"arer Ablauf & Der Benutzer fokusiert das Suchfeld und beginnt ein
        Suchwort einzugeben. Bei Eingabe jedes einzelnen Buchstabens wird die
        Tabelle darunter aktualisiert und nur noch Filme angezeigt, deren
        Informationen das Suchwort enthalten.\\
        \hline
        Nachbedingung & Angezeigt werden nur noch Filme, deren Informationen
        das Suchwort enthalten.\\
        \hline
        Alternative Abl\"aufe &
        \begin{enumerate}
          \item Kein Film enth\"alt Informationen mit dem Suchwort. Der Benutzer
          wird darauf hingewiesen.
        \end{enumerate}\\
        \hline
      \end{tabular}

    \subsection{Anwendungsfall A-06}\label{uc:06}

      \begin{tabular}{|b|p{10cm}|}
        \hline
        Name & Speichern der Sammlung \\
        \hline
        Aktoren & Benutzer \\
        \hline
        \hline
        Vorbedingung & Der Benutzer befindet sich in der Hauptansicht.\\
        \hline
        Regul\"arer Ablauf & Der Benutzer klickt auf ``Datei $\rightarrow$ Speichern'' und
        die Sammlung wird daraufhin abgespeichert.\\
        \hline
        Nachbedingung & Die Sammlung wurde gespeichert.\\
        \hline
        Alternative Abl\"aufe & Keine \\
        \hline
      \end{tabular}

    \subsection{Anwendungsfall A-07}\label{uc:07}

      \begin{tabular}{|b|p{10cm}|}
        \hline
        Name & Verleihen von Filmen \\
        \hline
        Aktoren & Benutzer \\
        \hline
        \hline
        Vorbedingung & Der Benutzer befindet sich in der Hauptansicht und es
        sind Filme in der Sammlung vorhanden.\\
        \hline
        Regul\"arer Ablauf & Der Benutzer selektiert einen Film in der Tabelle
        und klickt auf ``Verleihen''. Ein Dialog \"offnet sich, der Benutzer gibt
        Name und Datum ein und klickt auf ``Speichern''.\\
        \hline
        Nachbedingung & Der Film wird nun als Verliehen markiert und die
        Ausleihinformationen befinden sich in der Datenbank.\\
        \hline
        Alternative Abl\"aufe &
        \begin{enumerate}
          \item Der Benutzer hat keinen Film selektiert. Er wird darauf hingewiesen.
          \item Der Benutzer klickt im Dialog auf ``Abbrechen''. Die Daten
          werden nicht ver\"andert und er kehrt zur Hauptansicht zur\"uck.
          \item Der Benutzer gibt inkorrekte Daten ein. Eine Fehlermeldung
          wei{\ss}t ihn darauf hin.
        \end{enumerate}\\
        \hline
      \end{tabular}

    \subsection{Anwendungsfall A-08}\label{uc:08}

      \begin{tabular}{|b|p{10cm}|}
        \hline
        Name & Filterfunktion f\"ur verliehen Filme \\
        \hline
        Aktoren & Benutzer \\
        \hline
        \hline
        Vorbedingung & Der Benutzer befindet sich in der Hauptansicht und es
        sind mehrere Filme in der Sammlung vorhanden.\\
        \hline
        Regul\"arer Ablauf & Noch keine Ahnung.\\
        \hline
        Nachbedingung & Angezeigt werden nur noch Filme angezeigt die mindestens
        den angegeben Zeitraum verliehen wurden.\\
        \hline
        Alternative Abl\"aufe &
        \begin{enumerate}
          \item Kein Film ist bereits so lange verliehen. Der Benutzer wird
          darauf hingewiesen.
        \end{enumerate}\\
        \hline
      \end{tabular}

    \subsection{Anwendungsfall A-09}\label{uc:09}

      \begin{tabular}{|b|p{10cm}|}
        \hline
        Name & HTML-Export der Sammlung \\
        \hline
        Aktoren & Benutzer \\
        \hline
        \hline
        Vorbedingung & Der Benutzer befindet sich in der Hauptansicht und es
        sind Filme in der Sammlung vorhanden.\\
        \hline
        Regul\"arer Ablauf & Der Benutzer klickt auf ``Datei $\rightarrow$ Exportieren''
        und ein normaler Speicherndialog erscheint. Er w\"ahlt ein Verzeichnis
        und einen Dateinamen aus und klickt auf ``Speichern''.\\
        \hline
        Nachbedingung & In der angegebenen Datei befindet sich der HTML-Code
        zur Anzeige der Sammlung.\\
        \hline
        Alternative Abl\"aufe &
        \begin{enumerate}
          \item In der Sammlung befinden sich keine Filme. Der Benutzer wird
          darauf hingewiesen.
          \item Die angegebene Datei existiert bereits. Der Benutzer wird
          gefragt ob sie \"uberschrieben werden soll. Sollte er Nein w\"ahlen wird
          der Speicherndialog noch einmal angezeigt.
        \end{enumerate}\\
        \hline
      \end{tabular}

\pagebreak
\section{Anhang A}\label{sec:AnhangA}
  \subsection{Revisionshistorie}\label{sec:Revisionshistorie}

    \begin{tabular}{|b|l|}
    \hline
    Version 0.1, 23.02.2010 & Initiale Revision \\
    \hline
    \end{tabular}

\section{Anhang B}\label{sec:AnhangB}
  \subsection{Begriffslexikon}\label{sec:Begriffslexikon}

    \begin{tabular}{|b|l|}
    \hline
    Begriff & Beschreibung \\
    \hline
    aidGer & Aide Manager (zu deutsch: Hilfskraft Verwalter)
    \hline
    \end{tabular}    
\end{document}
