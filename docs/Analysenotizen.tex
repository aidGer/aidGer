%% LyX 1.6.5 created this file.  For more info, see http://www.lyx.org/.
%% Do not edit unless you really know what you are doing.
\documentclass[12pt,ngerman]{scrartcl}
\usepackage{ae,aecompl}
\usepackage[T1]{fontenc}
\usepackage[utf8]{inputenc}
\usepackage[letterpaper]{geometry}
\geometry{verbose}
\usepackage{babel}

\usepackage[unicode=true, pdfusetitle,
 bookmarks=true,bookmarksnumbered=false,bookmarksopen=false,
 breaklinks=false,pdfborder={0 0 1},backref=false,colorlinks=false]
 {hyperref}
\begin{document}

\title{Analysenotizen aidGer}

\author{Buchgraber (2512046), Gildein (2513744), Pirrung (2526016)\\
Gruppe 10}

\subject{- vertrauliches Dokument -}

\maketitle

\section*{Ist-Zustand}

    \subsection*{Prozess}
        \begin{itemize}
            \item Auf Papier definierte Abläufe und Geschäftsprozesse.
            \item Etabliertes und ausreichend gut funktionierendes Datenmodell.
            \item Die genauen Prozessskizzen werden uns noch zur Verfügung gestellt.
        \end{itemize}

    \subsection*{Komponenten}
        \begin{itemize}
            \item Ticket-System
            \item MySQL Datenbank
            \item Cronscripte mit SQL Befehlen
            \item OpenOffice.org Base Anwendung
            \item Java GUI Prototyp
        \end{itemize}

    \subsection*{Details zum Ticket-System}
        \begin{itemize}
            \item Früher: Internetinterface, mit dem man auf eMails antworten kann. 
            \item Tickets sollen manuell eingegeben werden können 
            \item Updates sollen nicht ständig per eMail geschrieben werden 
            \item Ablauf: Student (schickt eMail) $\rightarrow$ ruft Ticket-System
            auf $\rightarrow$ Benachrichtigung an beide Mitarbeiter per eMail
            (Antwort über Webinterface) $\rightarrow$ Student kriegt Antwort
            und beide Mitarbeiter auch. 
        \end{itemize}
        
    \subsection*{Gehaltsberechnung}
        \begin{itemize}
            \item Verschiedene Gehaltsstufen: ungeprüft, Bachelor, geprüfte Bachelor/Master 
            \item pro Monat einen Stundensatz pro Qualifikation 
            \item Stundensatz + Steuersatz 
        \end{itemize}
        Alle Geldangaben sind immer in Euro. 

    \subsection*{Verfügbarmachung des Ist-Zustandes}
        Spätestens Anfang nächster Woche sollten die momentanen Programme/Prozesse 
        im ILIAS bereitgestellt werden (Anmerkung: zT. bereits geschehen).

\section*{Soll-Zustand}

    \subsection*{Vorgangserfassung}
        Alle Vorgänge die getätigt werden sollen erfasst werden. \\
        Darunter fallen unter anderem:
        \begin{itemize}
            \item Arbeitsverträge, Budgetcheck, Lohnsteuerkarten. 
            \item Benachrichtigungen für Hilfskräfte (z.B. falls Lohnabrechnung verfügbar).
            Z.B. Liste mit allen Hilfskräften und Mails. 
            \item Protokoll soll alle Vorgänge auflisten können. 
        \end{itemize}

    \subsection*{AdoHive Interface}
        Die Zuständigkeit für die Datenbank liegt bei Team AdoHive. Wir erhalten
        von ihnen eine Schnittstellendokumentation. \\
        Die jetzige Datenbank wird dabei nicht großartig verändert und
        ein Zugriff auf diese eventuell über ILIAS verfügbar. \\
        \textbf{Datenbanktyp:} Apache Derby aka. JavaDB


    \subsection*{Berichtswesen}
        Folgende Berichte sollen sich erstellen lassen/automatisch generiert werden:
        \begin{itemize}
            \item Controlling - Interner Bericht 
            \item Auszug von HIWIS zu bestimmten Situationen 
            \item Prüfung von Lohnverteilung 
            \item Nachname - Vorname - Gehalt bis zu diesem Monat (Von Januar bis...) 
        \end{itemize}

    \subsection*{Budgetplanung}
        Das Budget wird vom Lehrplan am Anfang des Jahres vorgegeben sein. Es besteht
        aber die Möglichkeit, dass sich dieses im Laufe des Jahres ändert.

    \subsection*{Details zu Anwendung}

        Es soll nur eine Desktopanwendung sein, ohne sich irgendwo authentifizieren
        zu müssen \\
         Die Datensätze sollen parallel bearbeitbar sein, was jedoch selten
        vorkommen soll. Zum Großteil Aufgabe von Team AdoHive (Locking der
        Tabellen).

    \subsection*{Ausgaben}

        Die Ausgabe soll wie bisher als PDF erfolgen. Das bisherige Aussehen
        wird im ILIAS bereitgestellt. \\
        Die Berichte sollen auf jeden Fall optimiert werden zB. durch Angabe
        von Summen.

    \subsection*{Mitteilungen}
        \begin{itemize}
            \item Am Ende des Tages eine Vorgangsübersicht ausgeben 
            \item Bei Überschreitung der vorgegeben Stundenanzahl einer Hilfskraft 
        \end{itemize}

    \subsection*{Eingabe von Daten}
        \begin{itemize}
            \item Bei nicht ausgefüllten Feldern nicht unbedingt eine Meldung 
            ausgeben (z.B. nicht bei Bemerkung leer) 
            \item Zuordnung zu Hilfskräften, eventuell zu Veranstaltungen. Auch alte
            Vorgänge der Hilfskraft/Veranstaltung anzeigen. 
            \item Datum, Bearbeiter, Hilfskraft, Bemerkung, Sender, Adressat 
            \item Massenerfassung soll möglich sein 
        \end{itemize}

    \subsection*{Datenschutz}
        Hilskräfte, die zu lange gespeichert sind, werden anonymisiert.

    \subsection*{Zukunft}
        Datenaustauschformat mit Import/Export Funktion in der Applikation.

    \subsection*{Details zur Applikation}
        \begin{itemize}
            \item Implementierung mit Java 6 und Swing zur Oberflächengestaltung. 
            \item Sprache der Anwendung sollte Deutsch und nach Möglichkeit Englisch
            sein. Die Möglichkeit zur Nachrüstung weiterer Sprachen sollte gegeben
            sein. 
            \item Die Größe der Applikation sollte sich an Bildschirme zwischen 1600x1200
            und 1900x1200 Pixel richten. 
            \item Shortcuts für Funktionen wäre gut. zB. Enter um eine Eingabe zu beenden. 
            \item Die Einarbeitungszeit in die Applikation sollte möglichst gering sein. 
            \item Sie sollte Übersichtlich und einfach zu bedienen sein. 
            \item Die jetzige Lösung soll jederzeit als Fallback-Möglichkeit zur Verfügung
            stehen. D.h. das die neue Schnittstelle kompatibel zur alten bleiben
            muss. 
        \end{itemize}

    \subsection*{Programmierrichtlinien}
        Der Code soll gut kommentiert und leicht lesbar sein, um ihn leicht
        durch andere Personen erweiterbar zu halten. \\
        Die Benutzung von Code Conventions (Sun CC) wäre bevorzugt.


    \subsection*{Externe Pakete}
        Es soll eine noch genauer spezifizierte Version von iText benutzt werden
        um die PDF-Ausgabe zu realisieren. Zudem wird eine Version von AdoHive 
        durch Team AdoHive zur Verfügung gestellt. \\
        Weitere externe Bibliotheken müssen vom Betreuer genehmigt werden.

    \subsection*{Generell verfügbare Funktionen}
        Eine Hilfe und eventuell Drucken.

    \subsection*{Durchführung}
        \begin{itemize}
            \item Das Prozessmodell ist durch die Meilensteine vorgegeben. 
            \item Meinungs- oder Terminänderungen müssen in die Dokumente einfließen.
            \item Die Arbeit endet mit der Abgabe .
        \end{itemize}

\end{document}
